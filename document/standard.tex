\documentclass[12px]{report}

\usepackage[style=numeric,sorting=none,backend=biber]{biblatex}
\addbibresource{references.bib} % The file containing our references, in BibTeX format
\usepackage[colorlinks=true,linkcolor=violet]{hyperref}

\begin{document}

\chapter{Introduction}
\label{ch:intro}

This document establishes a standard for interoperability for Digital Health Applications that treat self-reported nicotine usage. It is based on the internationally recognized HL7\textsuperscript{\textregistered} FHIR\textsuperscript{\textregistered} (Fast Healthcare Interoperability Resources) standard for interoperability 
within healthcare \cite{FHIR}. This document is accompanied by FHIR artefacts which are helpful in implementing the standard. The FHIR project and artefacts can be found at the \href{https://simplifier.net/treat-nicotine-usage-diga}{FHIR project page} \cite{project}.
Henceforth in this text, the \textit{Standard} refers to this document together with the FHIR artefacts.

\section{Version History}
Make a nice table showing that this is v1 and it is in effect from March 2021

\section{Objective}
The purpose of this document is to establish a standard way of interoperability for Digital Health Applications (DiGAs) that treat self-reported nicotine usage (F17.2 using the ICD-10 coding system). The standard must be based on internationally recognized
standards for interoperability within healthcare. The specific use-cases that the Standard is targetting are:

\begin{itemize}
    \item Interoperability between similar DiGAs. DiGAs which treat self-reported nicotine usage are able to exchange an individual patient's data using the format established in the Standard.
    \item Interoperability between a DiGA and journalling systems. Electronic patient journals are able to extract information using the Standard on what treatment the patient has received through the DiGA and key insights gained by the DiGA on the patient and the patient's condition.
\end{itemize}

\section{Online presence and further development}
The FHIR resources developed for the Standard can be retrieved as a package at the \href{https://simplifier.net/treat-nicotine-usage-diga}{project page} \cite{project}, 
and the accompanying implementation guide which is a shorter version of this document can be found there too.

The source code for the Standard: this document as well as all FHIR resources and their implementation guide; are open source and can be 
found on \href{https://github.com/alex-therapeutics/diga-nicotine-usage-fhir}{GitHub} \cite{github}. Further development of the Standard
takes place there and anyone may raise issues or suggest changes to it which can be incorporated in future versions.
The project is maintained by Alex Therapeutics.

\section{Copyright and license}

\begin{itemize}
    \item The Standard, which is this document and the attached FHIR artefacts, is licensed under the Creative Commons Attribution 4.0 International License. 
To view a copy of this license, visit \href{http://creativecommons.org/licenses/by/4.0/}{http://creativecommons.org/licenses/by/4.0/} or send a letter to Creative Commons, PO Box 1866, Mountain View, CA 94042, USA.
Copyright for the Standard is held by Alex Therapeutics AB. 
    \item The source code for the Standard is licensed under the Apache 2.0 license. 
You can read a copy of this license in the \href{https://github.com/alex-therapeutics/diga-nicotine-usage-fhir/blob/main/LICENSE}{GitHub project} \cite{github}.
Copyright for the source code is held by Alex Therapeutics AB together with individual contributors to it.
    \item The base HL7\textsuperscript{\textregistered} FHIR\textsuperscript{\textregistered} standard which the Standard is derived from is copyrighted by HL7 and subject to their \href{http://hl7.org/fhir/license.html}{license} \cite{fhirlic}.
HL7, FHIR and the FHIR [FLAME DESIGN] are the registered trademarks of Health Level Seven International and their use does not constitute endorsement by HL7.
\end{itemize}

\chapter{Background}

The DiGA guide states that FHIR is the way to go

\section{FHIR}
short intro to FHIR

\chapter{Specification}

Here goes the profile

\section{Dependencies}
Explain that these profiles build on the core fhir spec and the basis-de spec.

\section{Artefacts}
The following profiles, extensions, valuesets and codesystems are part of this standard:

(list them)

You can find the implementation guide at (URI) and the released artifacts at (github releases) or (simplifier).

We will list the artifacts in question here along with figures describing them. Note that we will only show and discuss the differential tables here, the resources otherwise conform to the core FHIR specification and you can view the full resources in the implementation guide.

\subsection{NicotineReducingPatient}

etc.. one section for each thing

\printbibliography[heading=bibintoc, title={References}]

\chapter{Appendix}

Include a FHIR bundle resource example export

Include some JSON examples of each resource

\end{document}